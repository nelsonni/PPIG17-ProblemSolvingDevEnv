\documentclass{ppig}
\usepackage{epsfig}
\usepackage{booktabs}
\usepackage{ucs}
\usepackage[utf8x]{inputenc}

\title{Problem-Solving Applications in Developer Environments}

\author{Nicholas Nelson \\
  Electrical Engineering \& Computer Science \\
  Oregon State University \\
  nelsonni@oregonstate.edu}

\date{}

\begin{document}
\maketitle
\thispagestyle{empty}

\begin{abstract}
This file both describes and exemplifies the required format for papers submitted to PPIG 2016. Please review this file even if you have submitted to PPIG before as some formatting details have changed relative to previous years.
\end{abstract}

\section{Introduction}

This format is to be used for all PPIG 2016 paper submissions for review and camera-ready versions. The easiest way to do this is to replace the current content of this file with your own material.

The rest of this document describes the format you should use when preparing your submission.

\section{Preliminary Work}
Brief description of PPIG'17 short paper. Possible references to ICSME'17 and/or FSE'17 work?

\section{Research Approach}
Conducting user studies regarding interactions with modal cards-based IDE interfaces, in order to develop a feedback loop that continues to improve the tool as well as match it's capabilities to the problem-solving actions that programmers care about.

\section{Contributions}
Extension of previous work on Code Bubbles, Variolite, PatchWorks, and others. However, these works focused primarily on innovations in usage (interaction models for debugging in the case of Code Bubbles) or on particular tasks (code patch comparisons and navigational affordances in the case of PatchWorks, and exploratory coding in the case of Variolite).

Citation\cite{blackwell1999how}.

\section{Acknowledgements}

\bibliography{ppig-sample-bibliography}
\bibliographystyle{apacite} 
\end{document}
