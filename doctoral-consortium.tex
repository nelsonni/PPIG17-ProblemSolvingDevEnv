\documentclass{ppig}
\usepackage{epsfig}
\usepackage{booktabs}
\usepackage{ucs}
\usepackage[utf8x]{inputenc}

\title{Problem-Solving Applications in Developer Environments}

\author{Nicholas Nelson \\
  Electrical Engineering \& Computer Science \\
  Oregon State University \\
  nelsonni@oregonstate.edu}

\date{}

\begin{document}
\maketitle
\thispagestyle{empty}

\begin{abstract}
This file both describes and exemplifies the required format for papers submitted to PPIG 2016. Please review this file even if you have submitted to PPIG before as some formatting details have changed relative to previous years.
\end{abstract}

\section{Introduction}
Integrated Development Environments (IDEs), such as Eclipse and Visual Studio, consolidate developer toolsets into a single environment.
These environments strive for consistency in behavior and presentation, by adopting UI that accommodate different types of actions and behaviors and presenting.
The traditional UI for modern IDEs has centered around a bento-box style in which there is a lattice of different panes stretching across a single window.
The rationale for this design is one of practicality; maximizing the use of screen space to allow developers visibility into their code.
However, this rationale and accompanying UIs have led to a code-centric design that crowds out other important tasks that developers use when developing code.

- Introduce Programming as Problem-Solving and make citations to psych/CS literature regarding this phenomenon.

\section{Preliminary Work}
- Discuss current works regarding IDE UI in light of Programming as Problem-Solving.
- Brief description of PPIG'17 short paper. Possible references to ICSME'17 and/or FSE'17 work?

\section{Research Approach}
Conducting user studies regarding interactions with modal cards-based IDE interfaces, in order to develop a feedback loop that continues to improve the tool as well as match it's capabilities to the problem-solving actions that programmers care about.

\section{Contributions}
Extension of previous work on Code Bubbles, Variolite, PatchWorks, and others. However, these works focused primarily on innovations in usage (interaction models for debugging in the case of Code Bubbles) or on particular tasks (code patch comparisons and navigational affordances in the case of PatchWorks, and exploratory coding in the case of Variolite).

Citation\cite{blackwell1999how}.

\section{Acknowledgements}

\bibliography{ppig-sample-bibliography}
\bibliographystyle{apacite} 
\end{document}
